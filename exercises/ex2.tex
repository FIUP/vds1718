\newcommand{\lvar}[1]{\textrm{\normalfont{\texttt{lvar(#1)}}}}
\newcommand{\exTwo}{$if~\confSs{S}{s}\Rar{*}s'~then~\forall{} x \notin lvar\{S\}.s(x)=s'(x)$}
\newcommand{\exTwoM}[1]{$if~\confSs{#1}{s}\Rar{*}s'~then~\forall{} x \notin lvar\{#1\}.s(x)=s'(x)$}
\newcommand{\exTwoMM}[3]{$if~\confSs{#1}{#2}\Rar{*}#3~then~\forall{} x \notin lvar\{#1\}.#2(x)=#3(x)$}

\exercise{Esercizio 2}
{
	\begin{enumerate}
	\item Define formally a function $lvar:\stm\rightarrow\wp(\var)$ such that
	for any $S\in\stm$, \lvar{S} is the set of variables in \var{} that appear
	in the left-hand side of some assignment that occurs in the statement S.
	\item Prove that for any $S\in\stm, s,s'\in\states:$ \\
	\begin{align*}
	if \confSs{S}{s}\Rar{*}s' then \forall{} x \notin lvar\{S\}.s(x)=s'(x)
	\end{align*}
	\end{enumerate}
}
{
	\subsubsection{Parte 1}.
	Per definire la funzione \texttt{lvar} ne do una definizione
	composizionale. \\
	Gli elementi base sono:
	\begin{itemize}
	\item \lvar{\skipistr} = $\emptyset$
	\item \lvar{\memupdate{x}{a}} = \{x\}
	\end{itemize}
	Gli elementi composti invece:
	\begin{itemize}
	\item \lvar{\concat{$S_1$}{$S_2$}} = \lvar{$S_1$} $\cup$ \lvar{$S_2$}
	\item \lvar{\ifABC{b}{$S_1$}{$S_2$}} = \lvar{$S_1$} $\cup$ \lvar{$S_2$}
	\item \lvar{\wbS{b}{S}} = \lvar{$S$}
	\end{itemize}

	\subsubsection{Parte 2}. Per dimostrare \exTwo{} utilizzo l'induzione sulla struttura
	di $S\in\stm$ \\
	
	I casi base sono:
	\begin{itemize}
	\item \casobase{S=\skipistr}
	\casespace
	\\
	Se S=\skipistr{} allora comunque prenda 
	$s\in\stm$ ho che $\confSs{\skipistr}{s}\Rar{}s$. Per definizione di 
	\skipistr{} quindi lo stato s non viene modificato e quindi qualsiasi
	variabile non verrà alterata.

	\item \casobase{S=\memupdate{x}{a}}
	\casespace
	\\
	 Se S=\memupdate{x}{a} allora per
	definizione dell'update in memoria si ha che 
	$\confSs{\memupdate{x}{a}}{s}\Rar{}s[x\rightarrow\mathbb{A}[a]_s]=s'$ e
	quindi per definizione dell'update in memoria si ha che 
	$\forall{}y\in\var~.~y\notin\lvar{\memupdate{x}{a}}$ y non verrà aggiornata
	e quindi $s(y)=s'(y)$
	\end{itemize}


	I casi induttivi invece sono:
	\begin{itemize}
	\item \casoinduttivo{S=\concat{$S_1$}{$S_2$}}
    \casespace
	\\
	Devo dimostrare che 
	\begin{center}
	\exTwoM{\concat{$S_1$}{$S_2$}}
	\end{center}
	avendo come ipotesi induttive
	\begin{center}		
	\exTwoM{S_1} \\
	e\\
	 \exTwoM{S_2} 
	\end{center}
	(poichè $S_1$ e $S_2$ sono sottoprogrammi di $S$). 
	
	Suppongo per ipotesi che valga
	\[ \confSs{\concat{$S_1$}{$S_2$}}{s}\Rar{*}s'\footnote{se non termina lo
	statement diventa banalmente vero poichè l'ipotesi è falsa}, \]
    quindi
\[ 	\exists{}~k\in\mathbb{N} \quad \textrm{ tale per cui} \quad 
	\confSs{\concat{$S_1$}{$S_2$}}{s}\Rar{k}s' \]
	e quindi, per il \textbf{lemma di decomposizione}, 
	\[ 	\exists{}~k_0, k_1\in\mathbb{N},s''\in\states~tali~che~
	\confSs{S_1}{s}\Rar{k_0}s''~e~\confSs{S_2}{s''}\Rar{k_1}s'~con~k_o+k_1=k \]
	
	Poichè quindi 
	\[ \confSs{S_1}{s}\Rar{k_0}s'' \qquad \textrm{e} \qquad
	\confSs{S_2}{s''}\Rar{k_1}s' \] 
	per ipotesi induttiva valgono
\[ 	
	\forall{} x \notin lvar\{S_1\}.s(x)=s''(x)  \qquad \textrm{e} \qquad
	\forall{} x \notin lvar\{S_2\}.s''(x)=s'(x) 
	 \]
	Sia $y\notin\lvar{\concat{$S_1$}{$S_2$}}$  allora, per definizione di \texttt{lvar},
	\[  y\notin\lvar{$S_1$} \cup \lvar{$S_2$} \]
	e quindi in particolare 
\[ 	y\notin\lvar{$S_1$} \]
	allora per \hi vale
	$s(y)=s''(y)$. 
	Inoltre poichè
	\[ y\notin\lvar{$S_1$} \cup \lvar{$S_2$} \qquad \textrm{ si ha inoltre che } \qquad
	y\notin\lvar{$S_2$} \]
	e quindi per ipotesi induttiva 
	$s''(y)=s'(y)$. \\
	
	Concludendo quindi si ha che $s(y)=s''(y)=s'(y)$ e quindi $s(y)=s'(y)$.
	\postcasespace{}
	
	\item \casoinduttivo{S=\ifABC{b}{S_1}{S_2}}
	\casespace{}
	\\
	Devo dimostrare che
	\begin{center}
		\exTwoM{\ifABC{b}{S_1}{S_2}}
	\end{center}

	avendo come ipotesi induttive
	\begin{center}
	\exTwoM{S_1} e \exTwoM{S_2} 
	\end{center}
	(poichè $S_1$ e $S_2$ sono sottoprogrammi di
	$S$). 
	
	Suppongo per ipotesi che valga 
\[ 	\confSs{\ifABC{b}{S_1}{S_2}}{s}\Rar{*}s'\footnote{se non termina lo
	statement diventa banalmente vero poichè l'ipotesi è falsa}  \]
	Ho i seguenti 2	casi:
		\begin{itemize}
		\item\fbox{ $\mathbb{B}[b]_s=tt$}
		Nel caso in cui la valutazione della guardia nello stato $s$ si vera, allora 
		$\confSs{\ifABC{b}{S_1}{S_2}}{s}\Rar{}\confSs{S_1}{s}$ 
		sul quale so che
		vale l'ipotesi induttiva, poichè per ipotesi 
		\[ \confSs{\ifABC{b}{S_1}{S_2}}{s}\Rar{*}s' \]
		 di conseguenza 
		 \[ \exists~k\in
		\mathbb{N}~tale~che~\confSs{\ifABC{b}{S_1}{S_2}}{s}\Rar{k}s' \] 
		e quindi 
		\[ \confSs{\ifABC{b}{S_1}{S_2}}{s}\Rar{}\confSs{S_1}{s}\Rar{k-1}s' \] 
		
		Sia 
		\[ y\notin\lvar{\ifABC{b}{$S_1$}{$S_2$}} \]
		allora, per definizione di 
		\texttt{lvar}, 
		\[ y\notin\lvar{$S_1$} \cup \lvar{$S_2$} \]
		 e quindi in
		particolare 
		$y\notin\lvar{$S_1$}$. Per \hi vale quindi
		$s(y)=s'(y)$.

		\item \fbox{$\mathbb{B}[b]_s=ff$}: è analogo al precedente. 
		Nel caso in cui la
		valutazione della guardia nello stato $s$ si falsa, allora 
	\[ \confSs{\ifABC{b}{S_1}{S_2}}{s}\Rar{}\confSs{S_2}{s} \]
		sul quale so che
		vale l'\hi{}, poichè per ipotesi 
		\[ \confSs{\ifABC{b}{S_1}{S_2}}{s}\Rar{*}s'\]
		 di conseguenza 
		 	\[ \exists~k\in 	\mathbb{N}~tale~che~\confSs{\ifABC{b}{S_1}{S_2}}{s}\Rar{k}s' \]
		 e quindi 
	\[ 	\confSs{\ifABC{b}{S_1}{S_2}}{s}\Rar{}\confSs{S_2}{s}\Rar{k-1}s' \].
		 
		Sia $y\notin\lvar{\ifABC{b}{$S_1$}{$S_2$}}$ allora, per definizione di 
		\texttt{lvar}, $y\notin\lvar{$S_1$} \cup \lvar{$S_2$}$ e quindi in
		particolare $y\notin\lvar{$S_2$}$. Per ipotesi induttiva vale quindi
		$s(y)=s'(y)$.
		\end{itemize}

	\item \casoinduttivo{S=\wbS{b}{S_1}}
	\casespace
	\\
	\`{E} sufficiente fare l'unfolding e
	ricondursi al caso \texttt{if} già dimostrato in precedenza. 
	Infatti sia 
\[ 	s\in\states~tale~che~\confSs{\wbS{b}{S_1}}{s}\Rar{*}s' \]
	
	Per la definizione
	di \whileSOS{} ho che 
\[ 	\confSs{\wbS{b}{S_1}}{s}\Rar{}
	\confSs{\ifABC{b}{(\concat{$S_1$}{\wbS{b}{$S_1$}})}{\skipistr}}{s}\Rar{*}s' \]
	e quindi il caso \texttt{if} è già dimostrato.
	\end{itemize}
}

\newpage