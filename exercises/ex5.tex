\newcommand{\exFive}
{\concat{$(\wbS{b}{S})$}{$\ifABC{b}{S}{\skipistr}$}$~\cong{}_{sos}~\wbS{b}{S}$}

\newcommand{\exFiveIff}
{\confSs{\concat{$(\wbS{b}{S})$}{$\ifABC{b}{S}{\skipistr}$}}{s}$\Rar{*}{s'}~
\iff{}~\confSs{\wbS{b}{S}}{s}\Rar{*}{s'}$}

\newcommand{\exFiveLtR}
{\confSs{\concat{$(\wbS{b}{S})$}{$\ifABC{b}{S}{\skipistr}$}}{s}$\Rar{*}{s'}~
\Longrightarrow{}~\confSs{\wbS{b}{S}}{s}\Rar{*}{s'}$}

\newcommand{\exFiveRtL}
{$\confSs{\wbS{b}{S}}{s}\Rar{*}{s'}~\Longrightarrow{}~$
\confSs{
	\concat{$(\wbS{b}{S})$}{$\ifABC{b}{S}{\skipistr}$}}
	{s}\Rar{*}$s'$}

\newcommand{\exFiveR}[1]{\confSs{
	\concat
		{$(\wbS{b}{S})$}
		{$\ifABC{b}{S}{\skipistr}$}
		}{$s$}$~\Rar{#1}~s'$}

\exercise{Esercizio 5}
{
	Prove or disprove the following semantic equivalence:
	\begin{center}
	\exFive{}
	\end{center}
	where $b\in\bexp~and~S\in\stm$
}
{
	Per dimostrare:
	\begin{center}
	\exFive{}
	\end{center}
	posso dimostrare che:
	\begin{center}
	\exFiveIff{}
	\end{center}
	dimostrando i due versi dell'implicazione separatamente.\\

	\textbf{Implicazione $\Longrightarrow{}$}\\
	Devo dimostrare che:
	\begin{center}
	se \exFiveLtR{}
	\end{center}
	Dal momento che \exFiveR{*} allora $\exists~n~\in~\mathbb{N}~tale~che$\\
	\exFiveR{n}. Allora so che per il \textbf{lemma di decomposizione} che 
	$\exists{}~k_0,~k_1\in\mathbb{N},~s''\in\states~tali~che~n=k_0+k_1~e~
	\confSs{\wbS{b}{S}}{s}\Rar{k_0}s''~e~
	\confSs{\ifABC{b}{S}{\skipistr}}{s''}\Rar{k_1}s'$. \\
	Ma per il \textbf{lemma di terminazione} so che 
	\begin{center}
	se $\confSs{\wbS{b}{S}}{s}\Rar{*}s''~\Longrightarrow{}~\mathbb{B}[b]_s=ff$,
	\end{center}
	quindi 
	\begin{center}
	$\confSs{\ifABC{b}{S}{\skipistr}}{s''}\Rar{}
	\confSs{\skipistr}{s''}\Rar{}s''$
	\end{center}
	Quindi dopo il \texttt{while} lo stato $s$ non viene modificato.\\

	\textbf{Implicazione $\Longleftarrow{}$}\\
	\begin{center}
	se \exFiveRtL{}
	\end{center}
	Dal momento che per ipotesi $\confSs{\wbS{b}{S}}{s}\Rar{*}{s'}$ allora
	per il \textbf{lemma di terminazione} so che $\mathbb{B}[b]_s'=ff$.
	Inoltre, poichè $\confSs{\wbS{b}{S}}{s}\Rar{*}{s'}$, allora 
	$\exists~n~\in~\mathbb{N}~tale~che~\confSs{\wbS{b}{S}}{s}\Rar{n}{s'}$.
	Posso quindi applicare il \textbf{lemma di composizione} sulla parte destra
	dell'implicazione e quindi visto che:
	\begin{center}
	$\confSs{\wbS{b}{S}}{s}\Rar{n}{s'}$
	\end{center}
	Ottengo:
	\begin{center}
	\confSs{\concat{$(\wbS{b}{S})$}{$\ifABC{b}{S}{\skipistr}$}}{s}$\Rar{n}$
	$\confSs{\ifABC{b}{S}{\skipistr}}{s'}$
	\end{center}
	Ma per il \textbf{lemma di terminazione}, come detto precedentemente, so che
	$\mathbb{B}[b]_s'=ff$ e quindi:
	\begin{center}
	$\confSs{\ifABC{b}{S}{\skipistr}}{s'}\Rar{}\confSs{\skipistr}{s'}\Rar{}s'$
	\end{center}
	Terminando quindi nello stesso stato della parte sinistra dell'implicazione.
}
