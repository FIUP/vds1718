\newcommand{\stmminus}{\stm{}\textsuperscript{\textbf{-}}}

\exercise{Esercizio 6}
{
    Consider the sublanguage \stmminus{} of \stm{} where assignments are not allowed,that is,
    \begin{center}
    $\stmminus{} \ni{} S ::= \skipistr | S_1;S_2 | \ifABC{b}{S_1}{S_2} | \wbS{b}{S}.$
    \end{center}
    Prove that for any $S\in\stmminus{}$ and $s\in\states$, either $\Sds{}s=s$ or $\Sds{}s=undef$.
}
{
    Per dimostrare che vale questo che rimuovendo da \stm{} l'assegnamento
    allora comunque prenda uno statement questo non modificherà lo stato
    iniziale o ciclerà all'infinito. Per dimostrare questo posso dimostrare che 
    \begin{center}
    se $\Sds{}s~=~s'~\Longrightarrow~s=s'$
    \end{center} 
    Poichè posso vedere 
    \begin{center}
    $\Sds{}s=s$ or $\Sds{}s=undef$
    \end{center}
    Come
    \begin{center}
    $A~\lor~\neg{}B$
    \end{center}
    con
    \begin{center}
    $A~=~\Sds{}s=s$\\
    $\neg{}B~=~\Sds{}s=undef~\Longrightarrow~
    B~=~\exists{}s'\in\states.\Sds{}s=s'$
    \end{center}
    Quindi posso dimostrare
    \begin{center}
    se $B~\Longrightarrow~A$
    \end{center}
    Cioè
    \begin{center}
    se $\Sds{}s=s'~\Longrightarrow~\Sds{}s=s$
    \end{center}
    Che può essere visto, per determinismo, come
    \begin{center}
    se $\Sds{}s=s'~\Longrightarrow~s=s'$
    \end{center}
    Per dimostrare ciò procedo per induzione strutturale sui costrutti in 
    \stmminus{}.


    \casobase{S=\skipistr} In questo caso si ha che $\dsCtxt{\skipistr}s~
    =~(\idDS)s~=~s~\forall{}s\in\states$ quindi l'asserto è valido.

    \casoinduttivo{S=S_1;S_2} In questo caso si deve dimostrare che
    se $\dsCtxt{S_1;S_2}s=s'~\Longrightarrow~s=s'$ sapendo che per ipotesi
    induttiva:
    \begin{center}
    se $\dsCtxt{S_1}s=s'~\Longrightarrow~s=s'$
    \end{center}
    e:
    \begin{center}
    se $\dsCtxt{S_2}s=s'~\Longrightarrow~s=s'$
    \end{center}
    Per definizione della composizione so che 
    $\dsCtxt{S_1;S_2}~=~\dsCtxt{S_2} \circ \dsCtxt{S_1}$. Sia $s$ uno stato,
    e suppongo che $\dsCtxt{S_1}s=s'$, allora vale l'ipotesi induttiva e quindi
    ho che $\dsCtxt{S_1}s=s$ e quindi mi rimane $\dsCtxt{S_2}s$, sul quale,
    supponendo che $\dsCtxt{S_2}s=s'$, posso applicare l'ipotesi induttiva e
    ottenendo quindi $\dsCtxt{S_2}s=s$. Nel caso in cui 
    $\nexists{}s'\in\states.\dsCtxt{S_1}s=s'~o~\dsCtxt{S_1}s=s'$ allora si ha che il costrutto vale banalmente (ipotesi falsa).

    \casoinduttivo{S=\ifABC{b}{S_1}{S_2}} In questo caso si deve dimostrare che
    \\se $\dsCtxt{\ifABC{b}{S_1}{S_2}}s=s'~\Longrightarrow~s=s'$ sapendo che per
    ipotesi induttiva:
    \begin{center}
    se $\dsCtxt{S_1}s=s'~\Longrightarrow~s=s'$
    \end{center}
    e:
    \begin{center}
    se $\dsCtxt{S_2}s=s'~\Longrightarrow~s=s'$
    \end{center}
    Per definizione dell'\texttt{if} so che 
    $\dsCtxt{\ifABC{b}{S_1}{S_2}}~=~\cond{b}{\dsCtxt{S_1}}{\dsCtxt{S_2}}$. Sia
    $s$ uno stato, abbiamo 2 casi a seconda della valutazione della guardia $b$
    in $s$:
    \begin{itemize}
    \item Se $\evalBbs{b}{s}=tt$ allora $(\cond{b}{\dsCtxt{S_1}}{\dsCtxt{S_2}})s
    ~=~\dsCtxt{S_1}s$. Su $S_1$ posso applicare l'ipotesi induttiva, e quindi
    se $\dsCtxt{S_1}s=s'~\Longrightarrow~s=s'$
    
    \item Se $\evalBbs{b}{s}=ff$ allora $(\cond{b}{\dsCtxt{S_1}}{\dsCtxt{S_2}})s
    ~=~\dsCtxt{S_2}s$, Su $S_2$ posso applicare l'ipotesi induttiva, e quindi
    se $\dsCtxt{S_2}s=s'~\Longrightarrow~s=s'$
    \end{itemize}

    \casoinduttivo{S=\wbS{b}{S}} In questo caso si deve dimostrare che
    \\se $\dsCtxt{\wbS{b}{S}}s=s'~\Longrightarrow~s=s'$ sapendo che per
    ipotesi induttiva:
    \begin{center}
    se $\dsCtxt{S}s=s'~\Longrightarrow~s=s'$
    \end{center}
    Per definizione del \texttt{while} so che 
    $\dsCtxt{\wbS{b}{S}}~=~\fixp{F}$ con $F g = \cond{b}{g \circ \Sds}{\idDS}$.
    
    Per il \textbf{teorema di Knaster-Tarski} so che:
    \begin{center}
    $\fixp{F}~=~\unionsem{n}F^{n}\bot$
    \end{center}
    Quindi dimostro per induzione su $n$ che vale:
    \begin{center}
    se $(\unionsem{n}F^{n}\bot)s=s'~\Longrightarrow~s=s'$
    \end{center}

    \begin{itemize}
    \item \casobase{n~=~0} In questo caso abbiamo:
    \begin{center}
    $(F^{n}\bot)s~=~(F^{0}\bot)s~=~(\bot)s~=~undef$
    \end{center}
    Quindi ipotesi falsa
    \item \casoinduttivo{n~>~0} In questo caso dobbiamo dimostrare che vale:
    \begin{center}
    se $(F^{n}\bot)s=s'~\Longrightarrow~s=s'$
    \end{center}
    Sapendo che per ipotesi induttiva vale che:
    \begin{center}
    $\forall t\in\mathbb{N},~t<n,~se~(F^{t}\bot)s=s'~\Longrightarrow~s=s'$
    \end{center}
    Quindi:
    \begin{center}
    $(F^{n}\bot)s~=~(F(F^{n-1}\bot))s~=~(\cond{b}{F^{n-1}\bot \circ \Sds}
    {\idDS})s$
    \end{center}
    Quindi abbiamo 2 casi a seconda dalla valutazione di $b$ in $s$:
    \begin{itemize}
    \item Se $\evalBbs{b}{s}=ff$ allora $(\cond{b}{F^{n-1}\bot \circ \Sds}
    {\idDS})s~=~(\idDS)s~=~s$

    \item Se $\evalBbs{b}{s}=tt$ allora $(\cond{b}{F^{n-1}\bot \circ \Sds}
    {\idDS})s~=~(F^{n-1}\bot \circ \Sds)s$. Poichè per ipotesi induttiva so che
    se $\Sds{}s=s'~\Longrightarrow~s=s'$\footnote{Se l'ipotesi fosse falsa,
    allora tutto verrebbe vacuamente vero} allora:
    \[
        (F^{n-1}\bot \circ \Sds{})s~=~(F^{n-1}\bot)s
    \]
    Poichè $n-1~<~n$ allora posso utilizzare l'ipotesi induttiva su 
    $F^{n-1}\bot$ e quindi so che se \\$(F^{n-1}\bot)s=s'~\Longrightarrow~s=s'$
    \footnote{Se l'ipotesi fosse falsa, allora tutto verrebbe vacuamente vero}
    e quindi anche $(F^{n}\bot)s=s$
    \end{itemize}
    \end{itemize}
}
\newpage