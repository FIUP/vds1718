\newcommand{\stmminus}{\stm{}\textsuperscript{\textbf{-}}}

\exercise{Esercizio 6}
{
    Consider the sublanguage \stmminus{} of \stm{} where assignments are not allowed,that is,
    \begin{center}
    $\stmminus{} \ni{} S ::= \skipistr | S_1;S_2 | \ifABC{b}{S_1}{S_2} | \wbS{b}{S}.$
    \end{center}
    Prove that for any $S\in\stmminus{}$ and $s\in\states$, either $\Sds{}s=s$ or $\Sds{}s=undef$.
}
{
    Per dimostrare che vale questo che rimuovendo da \stm{} l'assegnamento
    allora comunque prenda uno statement questo non modificherà lo stato
    iniziale o ciclerà all'infinito. Per dimostrare questo posso dimostrare che 
    \[ \Sds{}s~=~s'~\Longrightarrow~s=s' \]
    Poichè posso vedere: 
    \[ \Sds{}s=s \quad or \quad \Sds{}s=undef \]
    Come
    \[ A~\lor~\neg{}B \]
    \[ con \qquad A~=~\Sds{}s=s \]
  \[  e~con \quad \neg{}B~=~\Sds{}s=undef \qquad ovvero \qquad5    B~=~\exists{}s'\in\states.\Sds{}s=s' \]
    Quindi dimostrare il seguente enunciato:
    \[ B~\Longrightarrow~A \]
    equivale a dimostrare: 
\[ \Sds{}s=s'~\Longrightarrow~\Sds{}s=s \]

    Che può essere visto, per determinismo, come
    \[ \Sds{}s=s'~\Longrightarrow~s=s' \]
    Per dimostrare ciò procedo per induzione strutturale sui costrutti in 
    \stmminus{}.

\begin{itemize}
	\item \casobase{S=\skipistr} \\
	In questo caso si ha che $\dsCtxt{\skipistr}s~
    =~(\idDS)s~=~s~\forall{}s\in\states$ quindi l'asserto è valido.

   \item \casoinduttivo{S=S_1;S_2} \\
    In questo caso si deve dimostrare che
    se $\dsCtxt{S_1;S_2}s=s'~\Longrightarrow~s=s'$ sapendo che per \textbf{ipotesi
    induttiva} valgono le seguenti:
 \begin{itemize}
    \item[IndH] $\dsCtxt{S_1}s=s'~\Longrightarrow~s=s'$
    \item[IndH] $\dsCtxt{S_2}s=s'~\Longrightarrow~s=s'$
 \end{itemize}
    Per definizione della composizione so che 
    \[ \dsCtxt{S_1;S_2}~=~\dsCtxt{S_2} \circ \dsCtxt{S_1} \]
    Sia $s$ uno stato,
    e suppongo che $\dsCtxt{S_1}s=s'$, allora \textbf{vale l'ipotesi induttiva} e quindi
    ho che $\dsCtxt{S_1}s=s$. 
    
    Mi rimane $\dsCtxt{S_2}s$, sul quale,
    supponendo che $\dsCtxt{S_2}s=s'$, posso applicare l'\textit{ipotesi induttiva} e
    ottenendo quindi $\dsCtxt{S_2}s=s$.
    
     Nel caso in cui 
    \[ \nexists{}s'\in\states.\dsCtxt{S_1}s=s'~o~\dsCtxt{S_1}s=s' \] allora si ha che il costrutto vale banalmente (ipotesi falsa).

    \item \casoinduttivo{S=\ifABC{b}{S_1}{S_2}} \\
    In questo caso si deve dimostrare che
    se \[ \dsCtxt{\ifABC{b}{S_1}{S_2}}s=s'~\Longrightarrow~s=s' \] sapendo che per
   \textbf{ ipotesi induttiva} valgono le seguenti:
    \begin{itemize}
    \item[IndH] $\dsCtxt{S_1}s=s'~\Longrightarrow~s=s'$
    \item[IndH] $\dsCtxt{S_2}s=s'~\Longrightarrow~s=s'$
	\end{itemize}
    Per definizione dell'\texttt{if} so che 
    \[ \dsCtxt{\ifABC{b}{S_1}{S_2}}~=~\cond{b}{\dsCtxt{S_1}}{\dsCtxt{S_2}} \]. 
    Sia
    $s$ uno stato, abbiamo 2 casi a seconda della valutazione della guardia $b$
    in $s$:
    \subitem[TRUE]  $\evalBbs{b}{s}=tt$ allora \[ (\cond{b}{\dsCtxt{S_1}}{\dsCtxt{S_2}})s
    ~=~\dsCtxt{S_1}s \]
    Su $S_1$ posso\textit{ applicare l'ipotesi induttiva}, e quindi
    \[  \dsCtxt{S_1}s=s'~\Longrightarrow~s=s' \]   
    \subitem[FALSE] Se $\evalBbs{b}{s}=ff$ allora \[ (\cond{b}{\dsCtxt{S_1}}{\dsCtxt{S_2}})s
    ~=~\dsCtxt{S_2}s \], Su $S_2$ posso \textit{applicare l'ipotesi induttiva}, e quindi
    se \[ \dsCtxt{S_2}s=s'~\Longrightarrow~s=s' \]
    \casoinduttivo{S=\wbS{b}{S}}\\
     In questo caso si deve dimostrare che
     \[ \dsCtxt{\wbS{b}{S}}s=s'~\Longrightarrow~s=s' \] sapendo che per
    \textbf{ipotesi induttiva }vale: 
    \[ \mathbf{[IndH]}\qquad \dsCtxt{S}s=s'~\Longrightarrow~s=s' \]

    Per definizione del \texttt{while} so che 
   \[  \dsCtxt{\wbS{b}{S}}~=~\fixp{F} \] \[ con \quad  \mathit{F}g = \cond{b}{g \circ \Sds}{\idDS} \].
 
    Per il \textbf{teorema di Knaster-Tarski} so che:
    \[ \fixp{F}~=~\unionsem{n}F^{n}\bot \]
    Quindi \textbf{dimostro per induzione} su $n$ che vale:
     \[ (\unionsem{n}F^{n}\bot)s=s'~\Longrightarrow~s=s' \]

    \begin{itemize}
    \item \casobase{n~=~0} \\
    In questo caso abbiamo:
\[     (F^{n}\bot)s~=~(F^{0}\bot)s~=~(\bot)s~=~undef \]
    Quindi ipotesi falsa(e conclusione banalmente vera).
  
    \item \casoinduttivo{n~>~0} \\
    In questo caso dobbiamo dimostrare che vale:
    \[ (F^{n}\bot)s=s'~\Longrightarrow~s=s' \]
    Sapendo che per \textbf{ipotesi induttiva} vale che:
    \[ \forall t\in\mathbb{N},~t<n,~se~(F^{t}\bot)s=s'~\Longrightarrow~s=s' \]
    Quindi:
   (\[ F^{n}\bot)s~=~(F(F^{n-1}\bot))s~=~(\cond{b}{F^{n-1}\bot \circ \Sds}
    {\idDS})s \]
    Quindi abbiamo 2 casi a seconda dalla valutazione di $b$ in $s$:

    \subitem[FALSE] Se $\evalBbs{b}{s}=ff$ allora \[ (\cond{b}{F^{n-1}\bot \circ \Sds}
    {\idDS})s~=~(\idDS)s~=~s \]
    \subitem[TRUE] Se $\evalBbs{b}{s}=tt$ allora \[ (\cond{b}{F^{n-1}\bot \circ \Sds}
    {\idDS})s~=~(F^{n-1}\bot \circ \Sds)s \]
    Poichè \textit{\textbf{per ipotesi induttiva}} so che: 
    \[ \mathbf{[IndH]}\qquad \Sds{}s=s'~\Longrightarrow~s=s' \]\footnote{Se l'ipotesi fosse falsa,
    allora tutto verrebbe vacuamente vero} allora:
    \[
        (F^{n-1}\bot \circ \Sds{})s~=~(F^{n-1}\bot)s
    \]
    Poichè $n-1~<~n$ allora posso utilizzare l'ipotesi induttiva su 
    $F^{n-1}\bot$ e quindi vale: \[ (F^{n-1}\bot)s=s'~\Longrightarrow~s=s' \]
    \footnote{Se l'ipotesi fosse falsa, allora tutto verrebbe vacuamente vero}
    e quindi anche $(F^{n}\bot)s=s$

    \end{itemize}
\end{itemize}
}
\newpage