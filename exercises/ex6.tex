\newcommand{\stmminus}{\stm{}\textsuperscript{\textbf{-}}}

\exercise{Esercizio 6}
{
    Consider the sublanguage \stmminus{} of \stm{} where assignments are not allowed,that is,
    \begin{center}
    $\stmminus{} \ni{} S ::= \skipistr | S_1;S_2 | \ifABC{b}{S_1}{S_2} | \wbS{b}{S}.$
    \end{center}
    Prove that for any $S\in\stmminus{}$ and $s\in\states$, either $\Sds{}s=s$ or $\Sds{}s=undef$.
}
{
    Per dimostrare che vale questo che rimuovendo da \stm{} l'assegnamento
    allora comunque prenda uno statement questo non modificherà lo stato
    iniziale o ciclerà all'infinito procedo per induzione strutturale sui
    costrutti in \stmminus{}.

    \casobase{S=\skipistr} In questo caso si ha che $\dsCtxt{\skipistr}s~
    =~(\idDS)s~=~s~\forall{}s\in\states$ quindi l'asserto è valido.

    \casoinduttivo{S=S_1;S_2} In questo caso si deve dimostrare che 
    $\forall~s\in\states~\dsCtxt{S_1;S_2}s=s$ oppure $\dsCtxt{S_1;S_2}s=undef$
    sapendo che per ipotesi induttiva:
    \begin{center}
    $\forall~s\in\states~\dsCtxt{S_1}s=s$ oppure $\dsCtxt{S_1}s=undef$
    \end{center}
    e:
    \begin{center}
    $\forall~s\in\states~\dsCtxt{S_2}s=s$ oppure $\dsCtxt{S_2}s=undef$
    \end{center}
    Per definizione della composizione so che 
    $\dsCtxt{S_1;S_2}~=~\dsCtxt{S_2} \circ \dsCtxt{S_1}$. Sia $s$ uno stato,
    poichè su $S_1$ vale l'ipotesi induttiva abbiamo 2 casi:
    
    \begin{itemize}
    \item Se $\dsCtxt{S_1}s=undef$ allora $(\dsCtxt{S_2} \circ \dsCtxt{S_1})s~=~
    undef$, poichè $undef$ composto qualsiasi cosa restituisce $undef$;

    \item Se $\dsCtxt{S_1}s=s$ allora $(\dsCtxt{S_2} \circ \dsCtxt{S_1})s~=~
    (\dsCtxt{S_2})s$ e per ipotesi induttiva so che $\dsCtxt{S_2}s$ può
    evolvere solamente in $s$ oppure $undef$. 
    \end{itemize}

    \casoinduttivo{S=\ifABC{b}{S_1}{S_2}} In questo caso si deve dimostrare che 
    $\forall~s\in\states$\\$\dsCtxt{\ifABC{b}{S_1}{S_2}}s=s$ oppure 
    $\dsCtxt{\ifABC{b}{S_1}{S_2}}s=undef$ sapendo che per ipotesi induttiva:
    \begin{center}
    $\forall~s\in\states~\dsCtxt{S_1}s=s$ oppure $\dsCtxt{S_1}s=undef$
    \end{center}
    e:
    \begin{center}
    $\forall~s\in\states~\dsCtxt{S_2}s=s$ oppure $\dsCtxt{S_2}s=undef$
    \end{center}
    Per definizione dell'\texttt{if} so che 
    $\dsCtxt{\ifABC{b}{S_1}{S_2}}~=~\cond{b}{\dsCtxt{S_1}}{\dsCtxt{S_2}}$. Sia
    $s$ uno stato, abbiamo 2 casi a seconda della valutazione della guardia $b$
    in $s$:
    \begin{itemize}
    \item Se $\evalBbs{b}{s}=tt$ allora $(\cond{b}{\dsCtxt{S_1}}{\dsCtxt{S_2}})s
    ~=~\dsCtxt{S_1}s$, ma su $S_1$ vale l'ipotesi induttiva, quindi 
    $\dsCtxt{S_1}s=s$ oppure $\dsCtxt{S_1}s=undef$
    
    \item Se $\evalBbs{b}{s}=ff$ allora $(\cond{b}{\dsCtxt{S_1}}{\dsCtxt{S_2}})s
    ~=~\dsCtxt{S_2}s$, ma su $S_2$ vale l'ipotesi induttiva, quindi 
    $\dsCtxt{S_2}s=s$ oppure $\dsCtxt{S_2}s=undef$
    \end{itemize}

    \casoinduttivo{S=\wbS{b}{S}} In questo caso si deve dimostrare che 
    $\forall~s\in\states~\dsCtxt{\wbS{b}{S}}s=s$ oppure 
    $\dsCtxt{\wbS{b}{S}}s=undef$ sapendo che per ipotesi induttiva:
    \begin{center}
    $\forall~s\in\states~\dsCtxt{S}s=s$ oppure $\dsCtxt{S}s=undef$
    \end{center}
    Per definizione del \texttt{while} so che 
    $\dsCtxt{\wbS{b}{S}}~=~\fixp{F}$ con $F g = \cond{b}{g \circ \Sds}{\idDS}$.
    
    Per il \textbf{teorema di Knaster-Tarski} so che:
    \begin{center}
    $\fixp{F}~=~\unionsem{n}F^{n}\bot$
    \end{center}
    Quindi dimostro per induzione su $n$ che vale:
    \begin{center}
    $\forall~s\in\states~(\unionsem{n}F^{n}\bot)s=s$ oppure 
    $(\unionsem{n}F^{n}\bot)s=undef$
    \end{center}
    \casobase{n~=~0} In questo caso abbiamo:
    \begin{center}
    $(F^{n}\bot)s~=~(F^{0}\bot)s~=~(\bot)s~=~undef$
    \end{center}
    \casoinduttivo{n~>~0} In questo caso dobbiamo dimostrare che vale:
    \begin{center}
    $\forall~s\in\states~(F^{n}\bot)s=s$ oppure $(F^{n}\bot)s=undef$
    \end{center}
    Sapendo che per ipotesi induttiva vale che:
    \begin{center}
    $\forall t\in\mathbb{N},~t<n,~\forall~s\in\states~(F^{t}\bot)s=s$ oppure
    $(F^{t}\bot)s=undef$
    \end{center}
    Quindi:
    \begin{center}
    $(F^{n}\bot)s~=~(F(F^{n-1}\bot))s~=~(\cond{b}{F^{n-1}\bot \circ \Sds}
    {\idDS})s$
    \end{center}
    Quindi abbiamo 2 casi a seconda dalla valutazione di $b$ in $s$:
    \begin{itemize}
    \item Se $\evalBbs{b}{s}=ff$ allora $(\cond{b}{F^{n-1}\bot \circ \Sds}
    {\idDS})s~=~(\idDS)s~=~s$

    \item Se $\evalBbs{b}{s}=ff$ allora $(\cond{b}{F^{n-1}\bot \circ \Sds}
    {\idDS})s~=~(F^{n-1}\bot \circ \Sds)s$, e quindi abbiamo 2 sottocasi a
    seconda della valutazione di $\Sds{}s$, che per ipotesi induttiva può
    evolvere solamente in $undef$ o $s$:
        \begin{itemize}
        \item Se $\Sds{}s~=~undef$ allora $(F^{n-1}\bot \circ \Sds)s~=~undef$

        \item Se $\Sds{}s~=~s$ allora $(F^{n-1}\bot \circ \Sds)s~=~
        (F^{n-1}\bot)s$ su cui posso applicare quindi l'ipotesi induttiva e
        quindi so che $(F^{n-1}\bot)s=s$ oppure $(F^{n-1}\bot)s=undef$ poichè
        $n-1<n$
        \end{itemize}
    \end{itemize}
}
\newpage