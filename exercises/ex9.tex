\exercise{Esercizio 9}
{
	Prove or disprove the following semantic equivalence:
	\begin{center}
	$\wbS{b}{S}\eqDS{}\wbS{b}{(\concat{$S$}{$\ifABC{b}{S}{\skipistr}$})}$
	\end{center}
	where $b \in{} \bexp{}$ and $S \in{} \stm{}$
}
{
	Per dimostrare che vale:
	\begin{center}
	$\wbS{b}{S}\eqDS{}\wbS{b}{(\concat{$S$}{$\ifABC{b}{S}{\skipistr}$})}$
	\end{center}
	Devo dimostrare che:
	\begin{center}
	$\dsCtxt{\wbS{b}{S}}=
	\dsCtxt{\wbS{b}{(\concat{$S$}{$\ifABC{b}{S}{\skipistr}$})}}$
	\end{center}
	Dove:
	\begin{center}
	$\dsCtxt{\wbS{b}{S}}=\fixp{F}$ con 
	$F~g~=~\cond{b}{g \circ{} \Sds{}}{\idDS}$ \\
	\end{center}
	e
	\begin{center}
	$\dsCtxt{\wbS{b}{(\concat{$S$}{$\ifABC{b}{S}{\skipistr}$})}}=
	\fixp{G}$ con \\
	$G~g~=~\cond{b}{g \circ{} \dsCtxt{\concat{$S$}
	{$\ifABC{b}{S}{\skipistr}$}}}{\idDS}~=~$ \\
	$=~\cond{b}{g \circ{} \dsCtxt{\ifABC{b}{S}{\skipistr}} \circ{} \Sds}
	{\idDS}~=$ \\
	$=~\cond{b}{g \circ{} \cond{b}{\Sds{}}{\skipistr} \circ{} \Sds{}}{\idDS}$
	\end{center}
	Quindi devo dimostare che vale:
	\begin{center}
	$\fixp{F}~=~\fixp{G}$
	\end{center}
	Per fare ciò posso dimostrare separatamente le due seguenti affermazioni:
	\begin{enumerate}
	\item $\fixp{F}~\mineq~\fixp{G}$
	\item $\fixp{G}~\mineq~\fixp{F}$
	\end{enumerate}

	\textbf{Prima parte $\fixp{G}~\mineq~\fixp{F}$}\\
	Per dimostrare che vale $\fixp{G}~\mineq~\fixp{F}$ utilizzo il
	\textbf{Fixed point induction lemma} che afferma che, date $f$ e $d$:
	\begin{center}
	se $fd~\mineq~d~\Longrightarrow~\fixp{f}~\mineq~d$
	\end{center}
	istanziato con:
	\begin{center}
	$f~=~G$ \\
	$d~=~\fixp{F}$
	\end{center}
	Quindi:
	\begin{center}
	se $G(\fixp{F})~\mineq~\fixp{F}~\Longrightarrow~\fixp{G}~\mineq~\fixp{F}$
	\end{center}
	Sviluppo quindi $G(\fixp{F})$:
	\begin{center}
	$G(\fixp{F})~=~\cond{b}{\fixp{F} \circ \cond{b}{\Sds{}}{\idDS} \circ{} 
	\Sds{}}{\idDS}$
	\end{center}
	Quindi dato uno stato $s$ si hanno due casi a seconda della valutazione
	della guardia b in $s$:
	\begin{itemize}
		\item \caso{\evalBbs{b}{s}=ff}: In questo caso si ha:
		\begin{center}
		$(\cond{b}{\fixp{F} \circ \cond{b}{\Sds{}}{\idDS} \circ{} \Sds{}}
		{\idDS})s~=~(\idDS{})s~=~s$
		\end{center}
		Ma quindi:
		\begin{center}
		$(\fixp{F})s~=~(F(\fixp{F}))s~=~(\cond{b}{\fixp{F} \circ{} \Sds{}}
		{\idDS})s~=$\\$=~(\idDS{})s~=~s$
		\end{center}
		Ottenendo difatto lo stesso stato.

		\item \caso{\evalBbs{b}{s}=tt}: In questo caso si ha:
		\begin{center}
		$(\cond{b}{\fixp{F} \circ \cond{b}{\Sds{}}{\idDS} \circ{} \Sds{}}
		{\idDS})s~=$\\$=~(\fixp{F} \circ \cond{b}{\Sds{}}{\idDS} \circ{} 
		\Sds{})s$
		\end{center}
		Abbiamo quindi due sottocasi a seconda della valutazione di $(\Sds{})s$:
		\begin{itemize}
			\item \caso{\Sds{}_s~=~undef}: in questo caso poichè ho:
			\begin{center}
			$(\fixp{F} \circ \cond{b}{\Sds{}}{\idDS} \circ{} \Sds{})s$
			\end{center}
			Allora tutto verrà valutato in $undef$ poichè la composizione di 
			$undef$ con qualsiasi cosa restituisce sempre $undef$.\\
			Dall'altra parte quindi troviamo:
			\begin{center}
			$(\fixp{F})s~=~(F(\fixp{F}))s~=~(\cond{b}{\fixp{F} \circ{} \Sds{}}
			{\idDS})s~=~(\fixp{F} \circ{} \Sds{})s~=~undef$
			\end{center}
			Poichè assumiamo: (i) $\evalBbs{b}{s}=tt$ e (ii) $\Sds{}_s~=~undef$

			\item \caso{\Sds{}_s~=~s'}: In questo caso si ha:
			\begin{center}
			$(\fixp{F} \circ \cond{b}{\Sds{}}{\idDS} \circ{} \Sds{})s~=~
			(\fixp{F} \circ \cond{b}{\Sds{}}{\idDS})s'$
			\end{center}
			che a seconda della valutazione di $b$ in $s'$ evolverà in:
			\begin{itemize}
			\item $(\fixp{F} \circ \Sds{})s'$ se $\evalBbs{b}{s'}=tt$
			\item $(\fixp{F} \circ \idDS)s'~=~(\fixp{F})s'~=~(F(\fixp{F}))s'~=~
			(\cond{b}{\fixp{F} \circ{} \Sds{}}{\idDS})s'~=~(\idDS{})s'~=~s'$ se 
			$\evalBbs{b}{s'}=ff$
			\end{itemize}
			Dall'altra parte quindi troviamo:
			\begin{center}
			$(\fixp{F})s~=~(F(\fixp{F}))s~=~(\cond{b}{\fixp{F} \circ{} 
			\Sds{}}{\idDS})s~=~(\fixp{F} \circ{} \Sds{})s~=~(\fixp{F})s'~=~
			(F(\fixp{F}))s'~=~(\cond{b}{\fixp{F} \circ{} \Sds{}}{\idDS})s'$
			\end{center}
			Che come nel caso precedente evolverà in:
			\begin{itemize}
			\item $(\fixp{F} \circ \Sds{})s'$ se $\evalBbs{b}{s'}=tt$
			\item $(\idDS{})s'~=~s'$ se $\evalBbs{b}{s'}=ff$
			\end{itemize}
		\end{itemize}
	\end{itemize}

	\textbf{Seconda parte $\fixp{F}~\mineq~\fixp{G}$}\\
	Per dimostrare che vale $\fixp{F}~\mineq~\fixp{G}$ utilizzo il
	\textbf{teorema di Knaster-Tarski}, che afferma che:
	\begin{center}
	$\fixp{F}~=~\unionsem{n}F^{n}\bot$
	\end{center}
	Quindi dimostro per induzione su $n$ che vale:
	\begin{center}
	$\unionsem{n}F^{n}\bot~\mineq~\fixp{G}$
	\end{center}

	\casobase{n~=~0} In questo caso abbiamo:
	\begin{center}
	$F^{n}\bot~=~F^{0}\bot~=~\bot$
	\end{center}
	E quindi per definizione di $\bot$ sappiamo che $\bot~\mineq~\fixp{G}$
	poichè $\bot$ è minore o uguale a qualsiasi altro elemento.

	\casoinduttivo{n~>~0} In questo caso dobbiamo dimostrare che vale:
	$F^{n}\bot~\mineq~\fixp{G}$, sapendo che, per ipotesi induttiva vale:
	\begin{center}
	$\forall~t\in\mathbb{N},~t<n$ vale $F^{t}\bot~\mineq~\fixp{G}$
	\end{center}
	Sviluppiamo quindi $F^{n}\bot$:
	\begin{center}
	$F^{n}\bot~=~F(F^{n-1}\bot)~=~\cond{b}{F^{n-1}\bot{} \circ \Sds{}}{\idDS}$
	\end{center}
	Quindi valutando il tutto in un certo stato $s$ abbiamo 2 casi a seconda
	della valutazione della guardia $b$.
	\begin{itemize}
		\item \caso{\evalBbs{b}{s}=ff} In questo caso si ha:
		\begin{center}
		$(\cond{b}{F^{n-1}\bot{} \circ \Sds{}}{\idDS})s~=~(\idDS)s~=~s$
		\end{center}
		Ma quindi:
		\begin{center}
		$(\fixp{G})s~=~(G(\fixp{G}))s~=~(\cond{b}{\fixp{G} \circ 
		\cond{b}{\Sds{}}{\idDS} \circ{} \Sds{}}{\idDS})s~=~(\idDS)s~=~s$
		\end{center}

		\item \caso{\evalBbs{b}{s}=tt} In questo caso si ha:
		\begin{center}
		$(\cond{b}{F^{n-1}\bot{} \circ \Sds{}}{\idDS})s~=~
		(F^{n-1}\bot{} \circ \Sds{})s$
		\end{center}
		Invece dall'altra parte ottengo:
		\begin{center}
		$(\fixp{G})s~=~(G(\fixp{G}))s~=$\\$=~(\cond{b}{\fixp{G} \circ 
		\cond{b}{\Sds{}}{\idDS} \circ{} \Sds{}}{\idDS})s~=$\\$=~
		(\fixp{G} \circ \cond{b}{\Sds{}}{\idDS} \circ{} \Sds{})s$
		\end{center}

		Quindi si hanno 2 sottocasi a seconda della valutazione di $(\Sds{})s$
		\begin{itemize}
			\item \caso{(\Sds{})s~=~undef} allora ottengo:
			\begin{center}
			$F^{n-1}\bot{} \circ undef~=~undef$
			\end{center}
			Poichè $undef$ composto con qualsiasi cosa ritorna sempre $undef$.\\
			E dall'altra parte:
			\begin{center}
			$\fixp{G} \circ \cond{b}{\Sds{}}{\idDS} \circ{} undef~=~undef$
			\end{center}
			Poichè come prima $undef$ composto con qualsiasi cosa ritorna
			sempre $undef$.

			\item \caso{(\Sds{})s~=~s'} allora si avrà:
			\begin{center}
			$(F^{n-1}\bot{} \circ \Sds{})s~=~(F^{n-1}\bot{})s'$
			\end{center}
			E dall'altra parte:
			\begin{center}
			$(\fixp{G} \circ \cond{b}{\Sds{}}{\idDS} \circ{} \Sds{})s~=$\\
			$=~(\fixp{G} \circ \cond{b}{\Sds{}}{\idDS})s'$
			\end{center}
			Quindi a seconda della valutazione di $b$ in $s'$ il tutto evolverà
			in:
			\begin{itemize}
			\item Se $\evalBbs{b}{s'}=ff$
			\begin{center}
			$(\fixp{G} \circ \cond{b}{\Sds{}}{\idDS})s'~=$\\
			$=~(\fixp{G} \circ \idDS)s'~=~(\fixp{G})s'~=~(G(\fixp{G}))s'~=$\\
			$=~(\cond{b}{\fixp{G} \circ \cond{b}{\Sds{}}{\idDS} \circ{} \Sds{}}
			{\idDS})s'~=$\\$=~(\idDS)s'~=~s'$
			\end{center}

			\item Se $\evalBbs{b}{s'}=tt$
			\begin{center}
			$(\fixp{G} \circ \cond{b}{\Sds{}}{\idDS})s'~=$\\
			$=~(\fixp{G} \circ \Sds{})s'$
			\end{center}
			che a seconda della valutazione di $(\Sds{})s'$ evolverà in $undef$
			oppure in $(\fixp{G})s''$
			\end{itemize}

			Dall'altra parte invece abbiamo che:
			\begin{center}
			$(F^{n-1}\bot{})s'~=~(F(F^{n-2}\bot{}))s'~=~
			(\cond{b}{F^{n-2}\bot{} \circ \Sds{}}{\idDS})s'$
			\end{center}
			Quindi anche in questo caso a seconda della valutazione di $b$ in
			$s'$ abbiamo 2 casi:
			\begin{itemize}
			\item Se $\evalBbs{b}{s'}=ff$
			\begin{center}
			$(\cond{b}{F^{n-2}\bot{} \circ \Sds{}}{\idDS})s'~=~(\idDS)s'~=s'$
			\end{center}
			come nel caso precedente
			\item Se $\evalBbs{b}{s'}=tt$
			\begin{center}
			$(\cond{b}{F^{n-2}\bot{} \circ \Sds{}}{\idDS})s'~=~
			(F^{n-2}\bot{} \circ \Sds{})s'$
			\end{center}
			e quindi se $(\Sds{})s'~=~undef$ il tutto evolverà a $undef$,
			altrimenti se $(\Sds{})s'~=~s''$ si avrà $(F^{n-2}\bot{})s''$\\
			A questo punto posso confrontare $\fixp{G}$ e $F^{n-2}\bot{}$ 
			poichè ho $n-2<n$ e quindi è possibile applicare l'ipotesi 
			induttiva che afferma che:
			\begin{center}
			$\forall~t\in\mathbb{N},~t<n$ vale $F^{t}\bot~\mineq~\fixp{G}$
			\end{center}
			Grazie a ciò posso quindi affermare che 
			$F^{n-2}\bot{}~\mineq~\fixp{G}$
			\end{itemize}
		\end{itemize}
	\end{itemize}
	Quindi per ogni caso dell'ipotesi induttiva vale 
	$F^{n}\bot{}~\mineq~\fixp{G}$
}
\newpage