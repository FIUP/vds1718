\newcommand{\repeatSbS}[3]{\texttt{repeat}~#1~\texttt{test}~#2~\texttt{then}~#3}

\newcommand{\PDef}{$\concat{$S_1$}{$\wbS{b}{(\concat{$S_2$}{$S_1$})}$}$}

\newcommand{\exThreeB}{$\repeatSbS{S_1}{b}{S_2}~\cong_{sos}~\PDef{}$}

\newcommand{\exThreeIff}{$
\confSs{\repeatSbS{S_1}{b}{S_2}}{s}\Rar{*}s'~\iff~
\confSs{\PDef{}}{s}\Rar{*}s'$}

\newcommand{\exThreeLtR}{$
\confSs{\repeatSbS{S_1}{b}{S_2}}{s}\Rar{*}s'~\Longrightarrow~
\confSs{\PDef{}}{s}\Rar{*}s'$}

\newcommand{\exThreeLtRn}[1]{$
\confSs{\repeatSbS{S_1}{b}{S_2}}{s}\Rar{#1}s'~\Longrightarrow~
\confSs{\PDef{}}{s}\Rar{*}s'$}

\newcommand{\exThreeRtL}{$
\confSs{\PDef{}}{s}\Rar{*}s'~\Longrightarrow~
\confSs{\repeatSbS{S_1}{b}{S_2}}{s}\Rar{*}s'$}

\newcommand{\exThreeRtLn}[1]{$
\confSs{\PDef{}}{s}\Rar{#1}s'~\Longrightarrow~
\confSs{\repeatSbS{S_1}{b}{S_2}}{s}\Rar{*}s'$}

\exercise{Esercizio 3}
{
	Assume that the language $\textbf{While}^{\textbf{+}}$ includes a new
	iterative command
	\begin{center}
	\repeatSbS{$S_1$}{$b$}{$S_2$}
	\end{center}
	where $b~\in~\bexp$ and $S_1,~S_2~\in~\stm$, whose informal semantics goes
	as follows:\\The following three steps are iteratively executed:
	\begin{enumerate}
	\item the command $S_1$ is executed
	\item the Boolean expression $b$ is evaluated; if $b$ evaluates to 
	\emph{false} then the loop is exited, otherwise the loop proceeds with step
	3
	\item command $S_2$ is executed
	\end{enumerate}
	\begin{itemize}
	\item [(a)] Define the small step operational semantics of this new
	iterative command.
	\item [(b)] Find a program \textbf{P} in the basic language \textbf{While} such that
	\begin{center}
	$\repeatSbS{S_1}{b}{S_2}~\cong_{sos}~P$
	\end{center}
	and formally prove this semantic equivalence.
	\end{itemize}
}
{
	\subsubsection{Parte (a)}
	\begin{center}
	\[ \confSs{\repeatSbS{S_1}{b}{S_2}}{s} \]
	\[~\Rar{[repeat_{sos}]}~
	\confSs{\concat{$S_1$}{\ifABC{b}{\concat{$S_2$}
	{(\repeatSbS{$S_1$}{$b$}{$S_2$})}}{\skipistr}}}{s} \]

	\end{center}
	L'idea di base è mimare esattamente quando descritto nella prima parte
	della consegna, eseguendo qualcosa di simile all'\emph{unfolding} della
	regola del while in semantica \texttt{SOS}. \\
	Per questo il comando
	\texttt{repeat} viene trasformato nella concatenazione del sottoprogramma
	$S_1$, che deve essere in ogni caso eseguito, con un \texttt{if}. 
	\begin{itemize}
		\item [FF] Nel caso
		la valutazione della guardia dell'\texttt{if} (la stessa guardia
		specificata nel comando del \texttt{repeat}) sia \textbf{falsa} nello stato $s$
		allora il \texttt{repeat} evolve in \skipistr{}.
		\item [TT] Viceversa, se la guardia
		viene valutata a \texttt{true}, allora il \texttt{repeat} evolve nella
		concatenazione tra il sottoprogramma $S_2$ e la ripetizione del 
		\texttt{repeat}, in modo tale da permettere la ripetizione iterativa.\\
	\end{itemize}
	

	\subsubsection{Parte (b)}
	\begin{center}
	$P~=~\PDef{}$
	\end{center}
	Per dimostrare che:
	\begin{center}
	\exThreeB{}
	\end{center}
	Dimostro che:
	\begin{center}
	\exThreeIff{}
	\end{center}
	Quindi posso dimostrare le due implicazioni del ``se e solo se''
	separatamente.


	\paragraph{IMPLICAZIONE: } \textbf{\Large{$\Longleftarrow$}} \\
	Devo dimostrare che:
	\begin{center}
	se \exThreeRtL{}
	\end{center}
	Dal momento che per ipotesi \[ \confSs{\PDef{}}{s}\Rar{*}s \] allora so che 
	\[ \exists{}~n~\in~\mathbb{N}~tale~che~\confSs{\PDef{}}{s}\Rar{n}s'  \]quindi
	posso dimostrare sulla lunghezza $n$ della derivazione che tale
	implicazione vale\footnote{Se non esistesse tale n allora avrei una sequenza di derivazione infinita e non terminerei mai in una configurazione finale}.\\
	\begin{itemize}
	\item\casobase{n=0} \\
	
	Questo caso è banalmente vero poichè $\confSs{\PDef{}}{s}$
	non è una configurazione finale, quindi non può evolvere in 0 passi in uno
	stato.

	\item\casoinduttivo{n>0} \\
	Devo dimostrare che 
	\begin{center}
	se \exThreeRtLn{n}
	 Dall'ipotesi induttiva: $ \quad\forall{}~k~\in~\mathbb{N}~.~k~<~n$ \\
	vale che se \exThreeRtLn{k}
	\end{center}
	Poichè, per ipotesi, $\confSs{\PDef{}}{s}\Rar{n}s'$ allora per il
	\textbf{lemma di decomposizione} 
	\[ \exists{}~k_0,~k_1~\in~\mathbb{N},~
	s''~\in~\states{}~tali~che~\confSs{S_1}{s}\Rar{k_0}s''~e~\confSs{
	\wbS{b}{(\concat{$S_2$}{$S_1$})}}{s''}\Rar{k_1}s'~con~k_0+k_1=n$  $\dagger  \]
	Ho quindi 2 casi a seconda della valutazione della guardia $b$ in $s''$ del
	\texttt{while}.

		\subitem caso \fbox{$\mathbb{B}[b]_{s''}=ff$} \\
		in questo caso si ha 
		
		\[ \confSs{\wbS{b}{(\concat{$S_2$}{$S_1$})}}{s''}\Rar{} \]
		\[ \confSs{\ifABC{b}{(\concat{\concat{$S_2$}{$S_1$}}
		{\wbS{b}{(\concat{$S_2$}{$S_1$})}})}{\skipistr}}{s''}\Rar{} \]
		\[ \confSs{\skipistr}{s''}\Rar{}s'' \]

		Quindi dall'altra parte dell'implicazione si ha:
		\[ \confSs{\repeatSbS{S_1}{b}{S_2}}{s}\Rar{} \]
		\[ \confSs{\concat{$S_1$}{\ifABC{b}{(\concat{$S_2$}
		{(\repeatSbS{$S_1$}{$b$}{$S_2$})})}{\skipistr}}}{s} \]

		Poichè per ipotesi sappiamo che
		\[  \confSs{S_1}{s}\Rar{k_0}s'' \] allora
		per il \textbf{lemma di composizione} posso affermare che

		\[ \confSs{\concat{$S_1$}{\ifABC{b}{(\concat{$S_2$}
		{(\repeatSbS{$S_1$}{$b$}{$S_2$})})}{\skipistr}}}{s}\Rar{k_0} \]
		\[ \confSs{
			\ifABC{b}{(\concat{$S_2$}{(\repeatSbS{$S_1$}{$b$}{$S_2$})})} 
			{\skipistr}}{s''} \]

		Per ipotesi inoltre sappiamo che $\mathbb{B}[b]_{s''}=ff$ quindi:
	
	\[ 	\confSs{
			\ifABC{b}{(\concat{$S_2$}{(\repeatSbS{$S_1$}{$b$}{$S_2$})})} 
			{\skipistr}}{s''}\Rar{} \]
	\[ 	\confSs{\skipistr}{s''}\Rar{}s'' \]

		Ottenedo lo stesso stato finale quindi dell'implicazione di partenza.

		\subitem caso $\mathbb{B}[b]_{s''}=tt$: in questo caso si ha 

		\[ \confSs{\wbS{b}{(\concat{$S_2$}{$S_1$})}}{s''}\Rar{} \]
		\[ \confSs{\ifABC{b}{(\concat{\concat{$S_2$}{$S_1$}}
		{\wbS{b}{(\concat{$S_2$}{$S_1$})}})}{\skipistr}}{s''}\Rar{} \]
	\[ 	\confSs{\concat{\concat{$S_2$}{$S_1$}}
		{\wbS{b}{(\concat{$S_2$}{$S_1$})}}}{s''} \]

		Poichè dal passo precedente $\dagger $sappiamo che 
		\[ \confSs{\concat{\concat{$S_2$}{$S_1$}}
			{\wbS{b}{(\concat{$S_2$}{$S_1$})}}}{s''}\Rar{k_1 -2}s' \]
		allora
		per il \textbf{lemma di decomposizione} \\ 
	\[ 	\exists{}~k_2,~k_3~\in~
		\mathbb{N},~s'''~\in~\states{}~tali~che~\confSs{S_2}{s''}\Rar{k_2}s'''\]
		\[\confSs{\concat{$S_1$}{\wbS{b}{(\concat{$S_2$}{$S_1$})}}}{s'''}
		\Rar{k_3}s'~con~k_2+k_3=k_1-2 \]
		Inoltre sappiamo che $k_3<n$ poichè 
	\[ 	k_3=k_1-2-k_2=n-k_0-k_2-2<n \]
		Dall'altra parte abbiamo invece che:
		\[ \confSs{\repeatSbS{S_1}{b}{S_2}}{s}\Rar{} \]
	\[ 	\confSs{\concat{$S_1$}{\ifABC{b}{(\concat{$S_2$}
		{(\repeatSbS{$S_1$}{$b$}{$S_2$})})}{\skipistr}}}{s}\Rar{k_0} \] (per il
		lemma di composizione)\\
	\[ 	\confSs{\ifABC{b}{(\concat{$S_2$}
		{(\repeatSbS{$S_1$}{$b$}{$S_2$})}{\skipistr}}}{s''} \]

		Per ipotesi sappiamo inotre che $\mathbb{B}[b]_{s''}=tt$ e che 
		$\confSs{S_2}{s''}\Rar{k_2}s'''$, quindi:
	
\[ 		\confSs{\ifABC{b}{(\concat{$S_2$}
		{(\repeatSbS{$S_1$}{$b$}{$S_2$})}{\skipistr}}}{s''}\Rar{} \]
	\[ 	\confSs{\concat{$S_2$}{(\repeatSbS{$S_1$}{$b$}{$S_2$})}}{s''}\Rar{k_2} \] per il lemma di composizione 

		\[ \confSs{\repeatSbS{S_1}{b}{S_2}}{s'''} \]

		A questo punto, sapendo che,
		$\confSs{\concat{$S_1$}{\wbS{b}{(\concat{$S_2$}{$S_1$})}}}{s'''}
		\Rar{k_3}s'$ con $k_3<n$, posso applicare l'ipotesi induttiva e affermare che:
	
		\[ \confSs{\repeatSbS{S_1}{b}{S_2}}{s'''}\Rar{*}s' \]
		
	\end{itemize}

	\paragraph{IMPLICAZIONE} \textbf{\Large{$\Longrightarrow$ }} 
	Devo dimostrare che:
	\begin{center}
	\exThreeLtR{}
	\end{center}
	
	Dal momento che 
	\[ \confSs{\repeatSbS{S_1}{b}{S_2}}{s}\Rar{*}s' \]
	 so che
	\[ \exists{}~n~\in~\mathbb{N}~tale~che~
	\confSs{\repeatSbS{S_1}{b}{S_2}}{s}\Rar{n}s' \]\footnote{Se n finito non esistesse allora avrei una sequenza di derivazione infinita e non arriverei mai ad uno stato termminale }
	Quindi posso dimostrare sulla
	lunghezza $n$ della derivazione che tale implicazione vale.

	\begin{itemize}
\item \casobase{n=0} \\
	Questo caso è banalmente vero poichè 
\[ \confSs{\repeatSbS{S_1}{b}{S_2}}{s} \]
	non è una configurazione finale,
	quindi non può evolvere in 0 passi in uno stato.
		
\item \casoinduttivo{n>0} \\
	 Devo dimostrare che 
	\begin{center}
	se \exThreeLtRn{n}
	\end{center}
	Avendo come ipotesi induttiva che 
	\[ \forall{}~k~\in~\mathbb{N}~.~k~<~n  \]vale che se \exThreeLtRn{k}
	Poichè, per ipotesi, \[ \confSs{\repeatSbS{S_1}{b}{S_2}}{s}\Rar{n}s' \] allora
	si ha che:

\[ 	\confSs{\repeatSbS{S_1}{b}{S_2}}{s}\Rar{}   \]
\[	\confSs{\concat{$S_1$}{\ifABC{b}{(\concat{$S_2$}
	{(\repeatSbS{$S_1$}{$b$}{$S_2$})})}{\skipistr}}}{s}\Rar{n-1}s \]

	Per il \textbf{lemma di decomposizione} so che 

\[ 	\exists{}~k_0,~k_1~\in~
	\mathbb{N},~s''~\in~\states{}~tali~che~\confSs{S_1}{s}\Rar{k_0}s'' \] e 
\[ 	\confSs{\ifABC{b}{(\concat{$S_2$}{(\repeatSbS{$S_1$}{$b$}{$S_2$})})}
	{\skipistr}}{s''}\Rar{k_1}s'~con~k_0+k_1=n-1 \]

	Ho quindi 2 casi a seconda della valutazione della guardia $b$ in $s''$ del
	\texttt{repeat}.
	

		\subitem caso $\mathbb{B}[b]_{s''}=ff$: in questo caso si ha 

		\[ \confSs{\ifABC{b}{(\concat{$S_2$}{(\repeatSbS{$S_1$}{$b$}{$S_2$})})}
		{\skipistr}}{s''}\Rar{} \]
		\[ \confSs{\skipistr}{s''}\Rar{}s'' \]

		Quindi dall'altra parte dell'implicazione si ha:
		\begin{center}
		$\confSs{\PDef{}}{s}\Rar{k_0}$ (per il lemma di composizione)\\
		$\confSs{\wbS{b}{(\concat{$S_2$}{$S_1$})}}{s''}$
		\end{center}
		Poichè per ipotesi sappiamo che $\confSs{S_1}{s}\Rar{k_0}s''$. Inoltre
		sempre per ipotesi abbiamo che $\mathbb{B}[b]_{s''}=ff$, quindi:
		\begin{center}
		$\confSs{\wbS{b}{(\concat{$S_2$}{$S_1$})}}{s''}\Rar{}$\\
		$\confSs{\ifABC{b}{(\concat{$S_2$}{(\concat{$S_1$}
		{\wbS{b}{(\concat{$S_2$}{$S_1$})}})})}{\skipistr}}{s''}\Rar{}$\\
		$\confSs{\skipistr}{s''}\Rar{}s''$
		\end{center}
		Ottenedo lo stesso stato finale quindi dell'implicazione di partenza.

		\subitem caso $\mathbb{B}[b]_{s''}=tt$: in questo caso si ha:

		\[ \confSs{\ifABC{b}{(\concat{$S_2$}{(\repeatSbS{$S_1$}{$b$}{$S_2$})})}
		{\skipistr}}{s''}\Rar{} \]
	
		\[ \confSs{\concat{$S_2$}{(\repeatSbS{$S_1$}{$b$}{$S_2$})}}{s''} \]

		Poichè per ipotesi so che $\confSs{\ifABC{b}{(\concat{$S_2$}
		{(\repeatSbS{$S_1$}{$b$}{$S_2$})})}{\skipistr}}{s''}\Rar{k_1}s'$ e 
		$\mathbb{B}[b]_{s''}=tt$ allora 
		$\confSs{\concat{$S_2$}{(\repeatSbS{$S_1$}{$b$}{$S_2$})}}{s''}
		\Rar{k_1-1}s'$ quindi posso applicare il \textbf{lemma di 
		decomposizione} e quindi so che
		\[ \exists{}~k_2,k_3\in{}\mathbb{N},~ s'''\in{}\states{}~tali~che~
		\confSs{S_2}{s''}\Rar{k_2}s'''~e~\confSs{\repeatSbS{S_1}{b}{S_2}}
		{s'''}\Rar{k_3}s'~con~k_2+k_3=k_1-1 \] \\
		Inoltre sappiamo che $k_3<n$ poichè $k_3=k_1-1-k_2=n-k_0-k_2-2<n$.\\
		Dall'altra parte abbiamo quindi:
	
		\[ \confSs{\PDef{}}{s}\Rar{k_0}\] (per il lemma di composizione)\\
		\[\confSs{\wbS{b}{(\concat{$S_2$}{$S_1$})}}{s''}\Rar{}\\ \]
		\[\confSs{\ifABC{b}{(\concat{\concat{$S_2$}{$S_1$}}
		{\wbS{b}{(\concat{$S_2$}{$S_1$})}})}{\skipistr}}{s''}\Rar{}\]
		\[\confSs{\concat{\concat{$S_2$}{$S_1$}}
		{\wbS{b}{(\concat{$S_2$}{$S_1$})}}}{s''}\]
		Per ipotesi inoltre so che $\confSs{S_2}{s''}\Rar{k_2}s'''$ quindi per
		il \textbf{lemma di composizione} ottengo:
	
		\[ \confSs{\concat{\concat{$S_2$}{$S_1$}}
		{\wbS{b}{(\concat{$S_2$}{$S_1$})}}}{s''}\Rar{k_2} \]
	\[ 	\confSs{\concat{$S_1$}{\wbS{b}{(\concat{$S_2$}{$S_1$})}}}{s'''} \]

		A questo punto posso quindi applicare l'ipotesi induttiva poichè $k_3<n$
		e quindi poichè \\$\confSs{\repeatSbS{S_1}{b}{S_2}}{s'''}\Rar{k_3}s'$
		allora 
		$\confSs{\concat{$S_1$}{\wbS{b}{(\concat{$S_2$}{$S_1$})}}}{s'''}
		\Rar{*}s'$

	\end{itemize}
}
\newpage