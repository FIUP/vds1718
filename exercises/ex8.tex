\newcommand{\and}{\wedge}

\newcommand{\exEightA}
{$ \wbS{b_1}{(\ifABC{b_2}{S}{skip})} $}

\newcommand{\exEightB}
{$  \wbS{(b_1 \and b_2)}{S} $}

\newcommand{\Sds}{\dsCtxt{S}}
\newcommand{\mineq}{\sqsubseteq}
\newcommand{\notmineq}{\not\sqsubseteq}

\exercise{Esercizio 8}
{
    Consider: 
    \begin{center}
    $ P_1 \equiv $ \exEightA \\
    $ P_2 \equiv $ \exEightB
    \end{center}
    where $ b_1, b_2 \in\bexp~and~S\in\stm$. Prove or disprove the following
    statements: 
    \begin{enumerate}
    \item $\dsCtxt{P_1} \mineq \dsCtxt{P_2}$
    \item $\dsCtxt{P_2} \mineq \dsCtxt{P_1}$
    \end{enumerate}
}
{
    Prima di tutto devo definire la semantica dei due programmi.\\
    \begin{center}
    $\dsCtxt{$\exEightA{}$}~=~\fixp{G}$ \\con 
    $G~g~=~\cond{b_1}{g \circ \cond{b_2}{\Sds}{\idDS}}{\idDS}$
    \end{center}
    e
    \begin{center}
    $\dsCtxt{$\exEightB{}$}~=~\fixp{F}$ \\con 
    $F~g~=~\cond{b_1\and{}b_2}{g \circ \Sds}{\idDS}$
    \end{center}
    
    \textbf{Prima parte $\dsCtxt{P_1} \mineq \dsCtxt{P_2}$}\\
    Devo dimostrare che $\dsCtxt{P_1} \mineq \dsCtxt{P_2}$ e quindi posso
    dimostrare che:
    \begin{center}
    $\fixp{G} \mineq \fixp{F}$
    \end{center}
    Per fare ciò utilizzo il \textbf{Fixed point induction lemma} che afferma
    che, date $f$ e $d$:
    \begin{center}
    se $fd~\mineq~d~\Longrightarrow~\fixp{f}~\mineq~d$
    \end{center}
    istanziato con:
    \begin{center}
    $f~=~G$ \\
    $d~=~\fixp{F}$
    \end{center}
    Quindi:
    \begin{center}
    se $G(\fixp{F})~\mineq~\fixp{F}~\Longrightarrow~\fixp{G}~\mineq~\fixp{F}$
    \end{center}
    Sviluppo $G(\fixp{F})$:
    \begin{center}
    $G(\fixp{F})~=~\cond{b_1}{\fixp{F} \circ \cond{b_2}{\Sds}{\idDS}}{\idDS}$
    \end{center}
    e $\fixp{F}$:
    \begin{center}
    $\fixp{F}~=~F(\fixp{F})~=~\cond{b_1\and{}b_2}{\fixp{F} \circ \Sds}{\idDS}$
    \end{center}
    Ragioniamo quindi per casi a seconda della valutazione di $b_1$ e $b_2$ su
    di un certo stato $s$:
    \begin{itemize}
        \item \caso{\evalBbs{b_1}{s}=ff} In questo caso, qualunque sia il
        valore di $\evalBbs{b_2}{s}$ si avrà che la prima parte evolverà in:
        \begin{center}
        $(\cond{b_1}{\fixp{F} \circ \cond{b_2}{\Sds}{\idDS}}{\idDS})s~=~(\idDS)s
        ~=~s$
        \end{center}
        La seconda, poichè $\evalBbs{b_1}{s}=ff$ anche
        $\evalBbs{b_1\and{}b_2}{s}=ff$ e quindi:
        \begin{center}
        $(\cond{b_1\and{}b_2}{\fixp{F} \circ \Sds}{\idDS})s~=~(\idDS)s~=~s$
        \end{center}

        \item \caso{\evalBbs{b_1}{s}=tt $ e $ \evalBbs{b_2}{s}=ff} Poichè 
        $\evalBbs{b_2}{s}=ff$ anche in questo caso si avrà che 
        $\evalBbs{b_1\and{}b_2}{s}=ff$ e quindi la prima parte evolverà in
        questo modo:
        \begin{center}
        $(\cond{b_1}{\fixp{F} \circ \cond{b_2}{\Sds}{\idDS}}{\idDS})s~=$\\
        $=~(\fixp{F} \circ \cond{b_2}{\Sds}{\idDS})s~=$\\
        $=~(\fixp{F} \circ \idDS)s~=~(\fixp{F})s~=~(F(\fixp{F}))s~=$\\
        $=~(\cond{b_1\and{}b_2}{\fixp{F} \circ \Sds}{\idDS})s~=~(\idDS)s~=~s$
        \end{center}
        E quindi dall'altra parte similmente avremo:
        \begin{center}
        $(\cond{b_1\and{}b_2}{\fixp{F} \circ \Sds}{\idDS})s~=~(\idDS)s~=~s$
        \end{center}

        \item \caso{\evalBbs{b_1}{s}=tt $ e $ \evalBbs{b_2}{s}=tt} In questo
        caso poichè entrambe le guardie vengono valutate a $tt$ allorà si avrà
        anche $\evalBbs{b_1\and{}b_2}{s}=tt$ e quindi la prima parte evolverà
        in:
        \begin{center}
        $(\cond{b_1}{\fixp{F} \circ \cond{b_2}{\Sds}{\idDS}}{\idDS})s~=$\\
        $=~(\fixp{F} \circ \cond{b_2}{\Sds}{\idDS})s~=$\\
        $=~(\fixp{F} \circ \Sds)s$
        \end{center}
        E dall'altra parte troviamo:
        \begin{center}
        $(\cond{b_1\and{}b_2}{\fixp{F} \circ \Sds}{\idDS})s~
        =~(\fixp{F} \circ \Sds)s$
        \end{center}
    \end{itemize}
    Quindi vale che $\dsCtxt{P_1} \mineq \dsCtxt{P_2}$

    \textbf{Seconda parte $\dsCtxt{P_2} \mineq \dsCtxt{P_1}$}\\
    In questo caso si ha che $\dsCtxt{P_2} \mineq \dsCtxt{P_1}$ è falsa ovvero
    $\dsCtxt{P_2} \notmineq \dsCtxt{P_1}$. Sia $s$ uno stato in cui:
    \begin{center}
    $\evalBbs{b_1}{s}=tt $ e $ \evalBbs{b_2}{s}=ff$
    \end{center}
    Allora si avrà che $P_2$ termina in un numero finito di passi senza
    modificare $s$:
    \begin{center}
    $(\fixp{F})s~=~(F(\fixp{F}))s~=~
    (\cond{b_1\and{}b_2}{\fixp{F} \circ \Sds}{\idDS})s~=~(\idDS)s~=~s$
    \end{center}
    Mentre $P_1$ non terminerà mai:
    \begin{center}
    $(\fixp{G})s~=~(G(\fixp{G}))s~=$\\$=~
    (\cond{b_1}{\fixp{G} \circ \cond{b_2}{\Sds}{\idDS}}{\idDS})s~=$\\$=~
    (\fixp{G} \circ \cond{b_2}{\Sds}{\idDS})s~=$\\$=~
    (\fixp{G} \circ \idDS)s~=$\\$=~
    (\fixp{G})s~=~(G(\fixp{G}))s~=~...$
    \end{center}
    
}
\newpage