\newcommand{\and}{\wedge}

\newcommand{\exEightA}
{$ \wbS{b_1}{(\ifABC{b_2}{S}{skip})} $}

\newcommand{\exEightB}
{$  \wbS{(b_1 \and b_2)}{S} $}

\exercise{Esercizio 8}
{
    Consider: 
    \begin{center}
    $ P_1 \equiv $ \exEightA \\
    $ P_2 \equiv $ \exEightB
    \end{center}
    where $ b_1, b_2 \in\bexp~and~S\in\stm$. Prove or disprove the following
    statements: 
    \begin{enumerate}
    \item $\dsCtxt{P_1} \mineq \dsCtxt{P_2}$
    \item $\dsCtxt{P_2} \mineq \dsCtxt{P_1}$
    \end{enumerate}
}
{
    Prima di tutto devo definire la semantica dei due programmi.\\
    
  \[   \dsCtxt{$\exEightA{}$}~=~\fixp{G} \]
    \[ G~g~=~\cond{b_1}{g \circ \cond{b_2}{\Sds}{\idDS}}{\idDS} \]
    e
    \[ \dsCtxt{$\exEightB{}$}~=~\fixp{F} \]
     \[ con \quad    F~g~=~\cond{b_1\and{}b_2}{g \circ \Sds}{\idDS} \]
    
    \subsubsection{Prima parte} 
    \[ \dsCtxt{P_1} \mineq \dsCtxt{P_2}\]
    Devo dimostrare che $\dsCtxt{P_1} \mineq \dsCtxt{P_2}$ e quindi posso
    dimostrare che:
    \[ \fixp{G} \mineq \fixp{F} \]
    Per fare ciò utilizzo il fixed point induction lemma: 
    \fil{}
 
    istanziato con:
    \begin{center}
    $f~=~G$ \\
    $d~=~\fixp{F}$
    \end{center}
    Quindi se \[ G(\fixp{F})~\mineq~\fixp{F}~\Longrightarrow~\fixp{G}~\mineq~\fixp{F} \]
   
    Sviluppo $G(\fixp{F})$:

    \[ G(\fixp{F})~=~\cond{b_1}{\fixp{F} \circ \cond{b_2}{\Sds}{\idDS}}{\idDS} \]
    e $\fixp{F}$:
    
    \[ \fixp{F}~=~F(\fixp{F})~=~\cond{b_1\and{}b_2}{\fixp{F} \circ \Sds}{\idDS} \]

    Ragioniamo quindi per casi a seconda della valutazione di $b_1$ e $b_2$ su
    di un certo stato $s$:
    \begin{itemize}
        \item \caso{\evalBbs{b_1}{s}=ff} In questo caso, qualunque sia il
        valore di $\evalBbs{b_2}{s}$ si avrà che la prima parte evolverà in:
        \[ (\cond{b_1}{\fixp{F} \circ \cond{b_2}{\Sds}{\idDS}}{\idDS})s~=~(\idDS)s
        ~=~s \]
       
        La seconda, poichè $\evalBbs{b_1}{s}=ff$ anche
        $\evalBbs{b_1\and{}b_2}{s}=ff$ e quindi:
       
       \[  (\cond{b_1\and{}b_2}{\fixp{F} \circ \Sds}{\idDS})s~=~(\idDS)s~=~ \]
        
        \item \caso{\evalBbs{b_1}{s}=tt $ e $ \evalBbs{b_2}{s}=ff} 
        
        Poichè 
        $\evalBbs{b_2}{s}=ff$ anche in questo caso si avrà che 
        $\evalBbs{b_1\and{}b_2}{s}=ff$ e quindi la prima parte evolverà in
        questo modo:

        \[ (\cond{b_1}{\fixp{F} \circ \cond{b_2}{\Sds}{\idDS}}{\idDS})s~= \]
       \[  =~(\fixp{F} \circ \cond{b_2}{\Sds}{\idDS})s~= \]
       \[  =~(\fixp{F} \circ \idDS)s~=~(\fixp{F})s~=~(F(\fixp{F}))s~= \]
       \[  =~(\cond{b_1\and{}b_2}{\fixp{F} \circ \Sds}{\idDS})s~=~(\idDS)s~=~s \]
       
        E quindi dall'altra parte similmente avremo:
       
        \[ (\cond{b_1\and{}b_2}{\fixp{F} \circ \Sds}{\idDS})s~=~(\idDS)s~=~s \]
   
        \item \caso{\evalBbs{b_1}{s}=tt $ e $ \evalBbs{b_2}{s}=tt} 
        
        In questo
        caso poichè entrambe le guardie vengono valutate a $tt$ allorà si avrà
        anche $\evalBbs{b_1\and{}b_2}{s}=tt$ e quindi la prima parte evolverà
        in:
        
        \[ (\cond{b_1}{\fixp{F} \circ \cond{b_2}{\Sds}{\idDS}}{\idDS})s~= \]
       \[  =~(\fixp{F} \circ \cond{b_2}{\Sds}{\idDS})s~= \]
        \[ =~(\fixp{F} \circ \Sds)s \]
       
        E dall'altra parte troviamo:
       
        \[ (\cond{b_1\and{}b_2}{\fixp{F} \circ \Sds}{\idDS})s~
        =~(\fixp{F} \circ \Sds)s \]
    
    \end{itemize}
    Quindi vale che $\dsCtxt{P_1} \mineq \dsCtxt{P_2}$

    \subsubsection{Seconda parte} 
    \[ \dsCtxt{P_2}  \mineq \dsCtxt{P_1} \]
    
    In questo caso si ha che $\dsCtxt{P_2} \mineq \dsCtxt{P_1}$ è falsa ovvero
    $\dsCtxt{P_2} \notmineq \dsCtxt{P_1}$. \\
    L'intuizione è che se in $s$ vale che:

    \[ \evalBbs{b_1}{s}=tt $ e $ \evalBbs{b_2}{s}=ff \]
   
    Allora si avrà che $P_2$ termina in un numero finito di passi senza
    modificare $s$:
    
    \[ (\fixp{F})s~=~(F(\fixp{F}))s~=~
    (\cond{b_1\and{}b_2}{\fixp{F} \circ \Sds}{\idDS})s~=~(\idDS)s~=~s \]
    
    Mentre $P_1$ non terminerà mai:
    \[ (\fixp{G})s~=~(G(\fixp{G}))s~= \]
    \[ =~(\cond{b_1}{\fixp{G} \circ \cond{b_2}{\Sds}{\idDS}}{\idDS})s~= \]
    \[ =~(\fixp{G} \circ \cond{b_2}{\Sds}{\idDS})s~= \]
    \[ =~(\fixp{G} \circ \idDS)s~= \]
    \[ =~ (\fixp{G})s~=~(G(\fixp{G}))s~=~... \]

    Suppongo per assurdo che valga $\dsCtxt{P_2}  \mineq \dsCtxt{P_1}$, allora
    per definizione di $\mineq$ ho che 
    \[
    \forall{}s,s'\in\states,~se~(\fixp{F})s=s'~\Longrightarrow~(\fixp{G})s=s'
    \]
    e suppongo che in $s$ valga 
    \[ \evalBbs{b_1}{s}=tt $ e $ \evalBbs{b_2}{s}=ff \]
    \\Allora abbiamo che $(\fixp{F})s=s(=s'~per~determinismo)$
    Se $(\fixp{G})s=s'$ allora per il \textbf{lemma di terminazione} so che
    $\evalBbs{b_1}{s'}=ff$ ma $(\fixp{F})s=s$ e quindi 
    $(\fixp{F})s~\neq~(\fixp{G})s$ ho un assurdo.
}
\newpage