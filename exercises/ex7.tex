\newcommand{\lub}{\bigsqcup}
\newtheorem{innercustomlem}{Lemma}
\newenvironment{customlem}[1]
{\renewcommand\theinnercustomlem{#1}\innercustomlem}
{\endinnercustomlem}

\exercise{Esercizio 7}
{
    Consider the CPO $\langle\wp(\mathbb{N}),\subseteq\rangle$ and the
    following map f: $\wp(\mathbb{N})\rightarrow\wp(\mathbb{N})$
    \begin{equation}
    f(S) =
    \begin{cases*}
    \emptyset & if S is finite \\
    \mathbb N        & otherwise
    \end{cases*}
    \end{equation}
    \begin{enumerate}
        \item Is f monotonic?
        \item Is f continuos?
    \end{enumerate}
    Formally prove the answer.
}
{
    Ricordiamo che:
    \begin{mydef}[Funzione monotona]
    %\hspace{\textwidth{}}
    Una funzione \textit{f} è \textbf{monotona} se, qualora valga una relazione
    d'ordine su due input, allora essa varrà anche sugli output. Nel nostra
    caso \textit{f} è monotona se vale \textit{A} $\subseteq$ \textit{B} 
    $\Rightarrow$ \textit{f(A)} $\subseteq$ \textit{f(B)}.
    \end{mydef}

    Per \textbf{dimostrare che f è monotona}, supponiamo che \textit{A} 
    $\subseteq$ \textit{B} ($\Rightarrow$ \textbar A\textbar{} $\leq$ \textbar B
    \textbar{}). Ci troviamo di fronte a due casi:
    \begin{enumerate}
        \item \textit{B} è finito;
        \item \textit{B} è infinito.
    \end{enumerate}

    Se vale (1), allora anche 
    \textbar A\textbar{} è finito e quindi \textit{f(A)} = \textit{f(B)} ($\iff$
    \textit{f(A)} $\subseteq$ \textit{f(B)} e \textit{f(A)} $\supseteq$ 
    \textit{f(B)} per la \textit{\textbf{proprietà anti-riflessiva del CPO}}).
    Ergo la condizione di monotonia è rispettata.\\

    Se vale (2), allora due sotto-casi si possono verificare:
    \begin{enumerate}
        \item A è finito $\Rightarrow$ \textit{f(A)} = $\emptyset$ $\subseteq$ 
        \textit{ f(B)} = $\mathbb{N}$.
        \item A è infinito $\Rightarrow$ \textit{f(A)} = \textit{f(B)} ($\iff$ 
        \textit{f(A)} $\subseteq$ \textit{f(B)} e \textit{f(A)} $\supseteq$ 
        \textit{f(B)} per la proprietà anti-riflessiva del CPO).
    \end{enumerate}

    Ergo, per ogni sotto-caso, la condizione di monotonia è rispettata.\\


    Per dimostrare che la funzione \textit{f}, monotona, su un CPO è 
    \textbf{continua} basta vedere se, per ogni catena non vuota 
    $Y\in\wp{(\mathbb{N})}$, vale che:
    \[\lub{} \{f(S) | S \in{}Y\} = f(\lub{}Y)\]
    Quindi posso dimostrare separatamente che:
    \begin{enumerate}
    \item $\lub{} \{f(S) | S \in{}Y\} \subseteq f(\lub{}Y)$
    \item $f(\lub{}Y) \subseteq \lub{} \{f(S) | S \in{}Y\}$
    \end{enumerate}

    (1) Per dimostrare che $\lub{} \{f(S) | S \in{}Y\} \subseteq{} 
    f(\lub{}Y)$, dal momento che $f$ è monotona e 
    $(\wp{\mathbb{N}}, \subseteq{})$ è un CPO posso appicare il 
    \textbf{Lemma 5.30}:\\
    \begin{customlem}{5.30}
    Siano $(D, \sqsubseteq)$ e $(D', \sqsubseteq{}')$ due CPO e 
    $f:D\rightarrow{}D'$, $f$ monotona, allora se $Y$ è una catena in $D$, 
    $\{fd~|~d\in{}Y\}$ è una catena in $D'$ e $\lub{}' \{f(S) | S \in{}Y\} 
    \sqsubseteq{}' f(\lub{}Y)$
    \end{customlem}
    Dal momento che quindi $f:\wp{(\mathbb{N})}\rightarrow{}\wp{(\mathbb{N})}$
    è monotona, $(\wp{(\mathbb{N})}, \subseteq)$ è un CPO, sia Y una catena
    qualsiasi in $\wp{(\mathbb{N})}$, allora vale che $\lub{}\{f(X)~|~X\in{}Y\}
    \subseteq{}f(\lub{}Y)$\\
    
    (2) Sia quindi Y una catena non vuota in $\wp{(\mathbb{N})}$. Allora il lub
    di Y è il più grande elemento in $Y$, che chiamiamo $X_0$. Allora:\\
    \hspace*{1.5cm}$f(\lub{}Y)~\subseteq~f(X_0)$ (per~definizione~di~$X_0$)\\
    \hspace*{1.5cm}$f(X_0)~\subseteq~\lub{}\{f(X)~|~X\in{}Y\}$ poiché $X_0\in{}
    Y$.\\
    Abbiamo quindi dimostrato che $f(\lub{}Y) \subseteq \lub{} \{f(S) | S
    \in{}Y\}$.\\ \\
    
    Unendo le dimostrazioni dei punti (1) e (2) si ha che $\lub{} \{f(S) | S
    \in{}Y\} = f(\lub{}Y)$ $\Rightarrow$ f è continua.

%    Applicando al nostro caso vedo che: 
%    $ \bigsqcup$ ' \{ \textit{f(S)} \textbar{} \textit{S} 
%    $\in\wp(\mathbb{N})$ \} = $\bigsqcup$ ' \{$\emptyset$, $\mathbb{N}$\} = 
%    $\mathbb{N} \]\ 
%    e che   \textit{f}($\bigsqcup\wp(\mathbb{N})$) = \textit{f}($\mathbb{N}$)= 
%    $\mathbb{N}$.\\
%    Ergo, la condizione di continuità è rispettata.\\
%
%    Inoltre, \textit{f} è \textit{strict} perchè sulla catena vuota vale che 
%    $\bot$ = \textit{f}($\bot$) che, nel nostro caso, equivale a $\emptyset$ = 
%    \textit{f}($\emptyset$)
}
\newpage